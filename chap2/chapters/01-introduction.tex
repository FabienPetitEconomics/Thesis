A recent literature has documented a decline in income and social mobility in the last decades of the 20th century that has strengthened the link between individuals' origins and their socio-economic outcomes; see, for example, \citet{Blanden2007Accounting} for the UK and \citet{Chetty2020Race} for the US.
%Understanding what has driven such a decline is therefore crucial to policy makers in order to reassert equality of opportunity. 
Existing work has proposed several explanations for the reduction in mobility, focusing, for example, on educational investments, non-cognitive skills, or the impact of geographical location, yet little attention has been paid to the role of the structure of employment. This is surprising given that the decrease in mobility has taken place roughly at the same time as labour markets in high-income economies witnessed an increase in employment polarization. Since the 1980s, the share in total employment of low- and high-paying occupations has increased at the expense of that of middling occupations,\footnote{See, for example, \citet{Autor2003Skill}, \citet{Autor2006Polarization}, \citet{Goos2007Lousy}, \citet{Dustmann2009Revisiting}, \citet{Goos2009Job}, and \citet{Cortes2016Middle} on the extent of polarization.} raising the question of whether individuals from less well-off backgrounds can still climb the social ladder as the middle rungs become scarce. 

This paper bridges the gap between the literature on social mobility and that on employment polarization. To do so, we depart from existing work in two respects. First, mobility is not defined in terms of income, as the literature tends to do;\footnote{While economists have tended to examine income mobility (e.g. \citealt{Blanden2007Accounting}, \citealt{Blanden2013Intergenerational}, \citealt{Chetty2014Land}), the literature on social mobility focuses on the analysis of socio-economic class. See \citet{Erikson1992Constant}, as well as \citet{Chan2007Class} and \citet{Erikson2010Social} for a discussion on social class and inter-generational mobility in the United-Kingdom.} rather we focus on occupations and define occupational categories in line with the employment polarization literature (see, for example, \citealt{Goos2014Explaining}). This allows us to identify whether the increased impact of parental income is being driven by how family background affects occupational outcomes. Second, while existing work on inter-generational mobility focuses on the correlation between parental characteristics and the outcomes of mature children, we argue that it is important to disentangle changes in mobility that are due to the \textit{intra-generational} component ---defined as the transition between the entry job and the job when mature--- from those due to the initial job that individuals hold. This change of emphasis allows us to examine whether the impact of parental income is correlated to changes in the structure of employment, thus raising the question of whether polarization has been one of the causes of the decline in mobility. 

We start our analysis by developing a simple model with two employment periods and three occupations, in which individuals differ both in parental income and innate ability. The latter is not initially observable by firms but may be observed at the end of the first period of employment. Crucially, we argue that different occupations have different informational contents. In particular, we suppose that while high-paying and middling jobs reveal the ability of the individual, low-paying jobs do not. 

We consider, for simplicity, only two levels of parental income such that when individuals are young those with low-parental income are initially randomly allocated to either low-paying or middling jobs, and those with high parental income to high-paying or middling jobs. In the second period, ability is revealed making those from poor households and with high ability switch occupations with those from rich households and with low ability. In this context, the availability of middling jobs is the key element determining the extent of mobility. With only a small number of middling jobs, the majority of those from low-income backgrounds will start their careers in low-paying jobs. Because only the few that are initially in middling jobs can reveal that they are of high ability, only a few individuals with low-income parents will be promoted into high-paying occupations and as a consequence few of those who started in high-paying occupations will be demoted. The result will be a lower degree of mobility than when middling jobs are plentiful.

The core of our analysis is an empirical assessment of occupational mobility which uses data from two mature British cohorts, the National Child Development Study (NCDS58) and the British Cohort Study (BCS70). The surveys cover individuals born in, respectively, 1958 and 1970 for whom we have full activity histories along with parental income. These data have been widely used to address the extent of mobility in the UK and existing work indicates that parent-child income mobility has declined for the younger cohort as compared to the older one.\footnote{See for example \citet{Blanden2007Accounting}, \citet{Nicoletti2007Intergenerational}, and \citet{Blanden2013Intergenerational}, as well as the work by sociologists such as \citet{Goldthorpe2007Intergenerational} and \citet{Erikson2010Social}.} Because we are interested in the structure of employment, we define four occupational categories, low-paying, middling and high-paying jobs, in line with the employment polarization literature, as well as a category including those out-of-work. The data also allows us to consider occupational outcomes both at the start of the individual's career\footnote{The ages at which interviews take place for the two cohorts are not identical. Both were interviewed at 42 years of age but differ in the ages of the earlier interviews, with those born in 1958 (resp. 1970) having an interview at age 23 (resp. 26). We use these ages to measure early-career occupations.} as well as when workers are mature, i.e. at age 42, and hence to consider occupations at different stages of the work-life.

Existing work on mobility has taken two approaches, either focusing on the correlation between the child's income or social status at around 40-years of age and that of the parent or examining lifetime dynamics independently of parental background.\footnote{See \citet{Jantti2015Income} for a review.} Our empirical framework aims to disentangle changes in social mobility that are due to the \textit{intra-generational} component ---defined as the transition between the entry job and the job when mature--- from those due to the \textit{inter-generational} component. We proceed in two steps, estimating first the impact of parental income on the child's first-period occupation and then the effect of first-period occupation on the occupation at age 42, as well as whether there is any remaining direct effect of parental income. We can hence ask whether the decline in mobility observed over the period is due to a greater impact of parental background on entry jobs or if the change has occurred mainly through differences in transition probabilities over the child's lifetime. 

Our focus is the comparison between the results for the 1958 cohort and those for the 1970 cohort. Our data indicate that the polarization that has been observed at the aggregate level also appears when we consider the employment structure for each cohort, indicating that those born in 1958 entered the labour market when middling jobs were plentiful, while those born in 1970 faced greater employment polarization. Moreover, the change in the structure of employment has been particularly marked regarding first-period occupations. To further understand the relationship between polarization and mobility, we measure both at the regional level. We consider differences in the impact of parental income on occupational outcomes (i.e. the degree of \emph{immobility}) across large regions, and construct for each region a measure of employment polarization for each cohort using data from the Labour Force Survey so as to compute the change in the extent of polarization faced by the two cohorts. This allows us to correlate the change in \emph{immobility} and the change in the share of middling employment at the regional level in order to ask whether these two variables have moved together. 

Our analysis provides three main results. The first concerns the fact that intra-generational mobility is an essential aspect of the observed correlation between parent and child outcomes. We find that for both cohorts individuals face a large likelihood of changing occupational category over their career. Notably, around 23\% and 30\%, respectively, of those initially in low-paying and middling occupations are in high-paying occupations when they are 42. In fact, for those two groups, less than half of those who were in each occupational category when young are in the same one as mature workers, with both the probabilities of moving upwards and downwards being large. Persistence is much higher for those starting in the best-paid jobs, but nevertheless, a third of them experience downwards mobility. Our results hence imply that it is important to understand career dynamics in order to explain the transmission of economic outcomes across generations.

Second, we find that the increased impact of family background on children's incomes identified in previous work also appears when we focus on occupations.\footnote{A few studies have considered occupational mobility, notably \citet{Long2013Intergenerational} who take a three-generation perspective, and \citet{Bell2018Land} who use recent British data. The occupational categories used are however not the same as those found in the employment polarization literature. For example,  \citet{Long2013Intergenerational} build four categories: white-collar, farmer, skilled and semi-skilled, and unskilled. \citet{Bell2018Land} use narrow occupational categories that they rank by median wages.} Moreover, the reduction in mobility is apparent at all the stages that determine an individual’s occupation when mature, as both the effect of parental income on first-period occupation and that on the job when mature controlling for initial occupation have become stronger for the younger cohort. These results raise the question of what are the implications of the disappearance of middling jobs for mobility. On the one hand, fewer individuals have access to those jobs when young, and those who do tend to come from better-off backgrounds; on the other, whether those in middling jobs move to high-paying occupations is more dependent on parental income for the younger than for the older cohort. The overall outcome are increased differences in intra-generational mobility according to family background. For those at the top of the parental-income distribution, upwards mobility during the working life has risen by about 5 percentage points, both for those starting in low-paid or middling jobs; in contrast it has declined by around 8 percentage points for those from less well-off families, irrespective of what job they initially held. That is, we observe that the possibility of career progression has become more dependent on parental background.

Lastly, when we exploit the regional dimension of our data we find a correlation between mobility and polarization which appears both over time and in the cross-section. At the individual level, our results indicate that the effect of parental income on occupational outcomes is stronger for individuals that---when young--- lived in areas with greater job polarization, indicating that a possible reason for the observed decline in mobility across the two cohorts is the disappearance of middling jobs.  
We then consider differences in \emph{immobility} across large regions and find that regions that have experienced a greater decline in the share of middling jobs are also those in which the impact of parental income has increased the most. These correlations are indicative that the disappearance of middling jobs may be one of the reasons behind the observed decline in mobility.

Our work is related to three strands of literature. 
First, it contributes to the literature on the determinants of inter-generational mobility which has extensively documented the parent-child dynamics in income and social class.\footnote{See, for example, \citet{Nicoletti2007Intergenerational}, \citet{Kopczuk2010Earnings}, \citet{Blanden2013Intergenerational}, \citet{Long2013Intergenerational}, and  \citet{Chetty2014United}, \citet{Chetty2017Fading} for work on inter-generational income mobility and \citet{Erikson1992Constant}, \citet{Chan2007Class}, \citet{Goldthorpe2007Intergenerational}, and \citet{Erikson2010Social} on social class.} Much of the focus has been on how individual characteristics affect income dynamics across generations, notably education,  non-cognitive skills and personality traits, and the quality of the neighborhood.\footnote{See \citealt{Bjorklund2012Important}, \citealt{Blanden2014Education}, \citealt{Blanden2016Educational}, \citealt{Crawford2016Higher}, and \citealt{Neidhofer2018Educational} on education, \citealt{Chetty2020Race} ) on race, and \citealt{Heckman2006Effects}, \citealt{Blanden2007Accounting}, \citealt{Heckman2013Understanding}, and \citealt{Chetty2014Land}) on other childhood outcomes.} Yet little attention has been paid to the importance of early labour market experiences. This paper hence provides a bridge between the literatures on \textit{inter-generational} and \textit{intra-generational} mobility by focusing on access to jobs at the beginning of the career and the subsequent career dynamics, and shows that understanding \textit{intra-generational} mobility is essential to understand an individual's outcome when mature. 

Our paper is particularly close to the recent literature that has identified a reduction in income mobility and an increased role of parental background, notably in the US and the UK. Part of this effect seems to operate through education. For example, for the UK, \citet{Blanden2004Family} and \citet{Gregg2010Family} find a rising impact of parental income on children's educational attainment. More recent work, such as \citealt{Chetty2014United} has shown the importance of the location where the individual grew up for inter-generational income dynamics. Our contribution lies in showing that the increased importance of parental income also appears when we focus on occupational categories, and that this operates in part through a stronger influence of family background on the probabilities of moving from one occupation to another.

Lastly, our paper adds to our understanding of the consequences of employment polarization. Much of this literature has used search models and taken a macroeconomic approach to understand the causes and consequences of polarization. The role of routine-biased technological change has been at the center of the debate. Starting with \citet{Autor2003Skill}, these analyses maintain that advances in information and communication technology affected the structure of employment because the tasks that computers are good at performing are concentrated around a set of middle skills that have a considerable ``routine'' component.\footnote{See also \citet{Goos2014Explaining}, \citet{Caines2017Complex}, \citet{Lordan2018People}, and \citet{Acemoglu2020Robots}, i.a..} The tasks approach which assigns skills to tasks based upon comparative advantage has been fruitful in creating a framework that allows us to understand the allocation of labour both within and across countries, and its implications for the structure of employment and wages. Our paper departs from this literature by proposing a model based on the idea that tasks also differ in the possibilities they give individuals to transmit information to firms about their (initially) unobservable ability. As a result, different occupations will have a different potential to reveal individuals' skills, with important implications for occupational dynamics.

Concerning the consequences of polarization, economists have mainly focused on the distribution of earnings,\footnote{This literature has grown rapidly over the past decade. See, amongst others, \citet{Autor2013Growth}, \citet{Beaudry2016Great}, \citet{Caines2017Complex}, \citet{Ross2017Routine}, \citet{Barany2018Job}. and \citet{Longmuir2020Routinization}.} although there is some work on its impact on educational attainment or the labour supply  (\citealt{Spitz-Oener2006Technical}; \citealt{Verdugo2020Labour}). The task approach introduced by \citet{Autor2003Skill} implies that biased technological change results in both the polarization of employment and a change in wages, and much work has been devoted to trying to understand to what extent polarization has driven observed increases in earnings inequality.\footnote{The widespread view is that indeed the changing structure of employment has resulted in increased earnings dispersion; see the overview in \citet{Acemoglu2011Skills}. Some authors nevertheless disagree; see \citet{Hunt2019Employment}.} Surprisingly, the question of whether employment polarization affects mobility has been largely ignored. To our knowledge, the only exception is \citet{Hennig2021Labor}, who examines the relationship between the structure of employment and income mobility. He builds a model in which the disappearance of routine jobs results in a polarization of education and lower inter-generational mobility, predictions that are shown to be consistent with patterns of inter-generational income mobility in the US. In his framework, the occupation of mature workers is determined exclusively by their educational choice; we hence complement his work by adding an analysis of job-to-job transitions. We show that these transitions are essential to understand mobility and, when we turn to regional data, that they are correlated with the increase in polarization across cohorts. 

The paper is organised as follows. We start by presenting a simple model of the effect of employment polarization on occupational mobility. Section \ref{chap2-data} presents the cohort data and describes the structure of employment for the two cohorts along with their occupational dynamics. We discuss our empirical specification in Section \ref{chap2-specification}, which distinguishes between the effect of parental income on initial occupations and on the transition across occupations during the individual's worklife. Section \ref{chap2-mobility} focuses on the patterns of occupational mobility, examining the changes that have occurred across the two cohorts. The final step in our analysis, provided in Section \ref{chap2-regional}, is to estimate mobility at the regional level and provide evidence on the correlation across regions between the extent of job polarization and changes in mobility. Section \ref{chap2-conclusion} concludes.
