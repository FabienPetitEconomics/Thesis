A vast literature has discussed the consequences of job polarization for wage inequality. In contrast, little is known about whether the change in the employment structure has also had an impact on social mobility. This paper raises such question using British data for two cohorts for which we have information for parents and children.

We start by developing a simple theoretical setup with three types of jobs and two levels of parental income. Parental background will affect the child's human capital so that the latter’s productivity is determined both by human capital and innate (and initially unobservable) ability. Children’s entry jobs will be determined by parental background, but as their ability is revealed, they may move up or down the occupational ladder. As the share of middle jobs disappears, the possibilities for mobility fall, thus leading to greater job persistence across generations.

The model highlights not only the importance of polarization for social mobility, but also the fact that transitions across occupations---i.e. intra-generational occupational dynamics---are an essential aspect of inter-generational mobility. Our empirical approach uses data on two British cohorts that are particularly suited for our purposes. First, the two cohorts, born 12  years apart, entered the labour market under substantially different conditions in terms of the structure of employment, with the latter cohort facing a much more polarized labour market. Second, we have data for children at various ages so that we can identify to what extent upwards mobility is driven by an improvement in the occupation at which children enter the labour market or by them going up the occupational ladder during their work-life. 

The data indicate that intra-generational occupational changes are an important source of mobility, with large shares of those starting in low-paying and middling occupations moving, respectively, to middling and high-paying jobs over their work lives. When we compare the two cohorts, we find that as the share of middling jobs has fallen these two sources of occupational mobility have weakened. Our results indicate that the role of parental income in determining occupations has increased, both for first-period occupations and for the transition towards better-paid occupations. For example, the fortunes of those who start in low-paying jobs differ considerably across generations. For the older cohort, a considerable fraction moved into middling jobs, but this probability has fallen markedly for the younger cohort. At the same time, the probability for those who start in low-paying jobs to move into high-paying jobs has remained roughly stable on average, but this average hides the fact that it has considerably increased for those with high-income parents and declined for those from low-income backgrounds. 

Although our data does not allow us to establish causality, the changes we identify are suggestive that as middling jobs have been eroded, parental income has become more important in determining occupational outcomes. Our analysis of regional mobility patterns finds that regions where employment polarization rose the most are also those where \emph{immobility} increased the most. These results hence suggest that the structure of employment affects not only the distribution of earnings but also the degree of occupational mobility. Moreover, they point towards the possibility that there is a transmission of polarization across generations, and that the increased importance of parental background may accumulate across generations creating a multiplier effect that over time accentuates the occupational distance across groups from different backgrounds. This is a question that we intend to pursue in future work. 