An extensive literature has studied the effect of life experiences on values but supposing that values are independent. I present a framework that jointly analyzes the dynamics of values over the lifecycle when life events provide information shocks on values in a context where values are inter-dependent in society. My results suggest that values inter-dependence plays an important role as individuals seek to be consistent with respect to values held in the group with which they identify. Thus, neglecting this mechanism underestimates to which extent life experiences affect individuals.

This paper has two main limitations which relate to the theoretical framework. First, I assume that value frontiers between groups are exogenous, while they are most likely endogenous. In my theoretical framework, I assume that the population is sufficiently large to ensure the anonymity of the agent, meaning that any change of value from the agent does not change the distribution. Relaxing this hypothesis would make the value frontier between groups endogenous. It would also relate the theoretical framework to the literature on network. Considering, for instance, that some individuals are more influential than others according to their position within the network. Such a framework could lead to a new approach in linking behaviors, values, and networks in a context of inter-dependence between values. Although I do not consider this approach in this paper, I intend to explore it in future works.

Second, I focus on individual life events, hence, the model is a partial equilibrium model. Thus, I suppose that values held in the group are time-invariant. An extension of the model would be to make them time-dependent, hence, sufficiently large shocks in one period, such as economic crises or global pandemics, would affect the average values. However, this extension goes beyond the scope of the paper and is also intentionally left for future research.

This paper raises an issue that has not been considered in the economic literature yet, namely, the consequences of life events on values. As values are at the roots of agents' preferences---which themselves can explain gaps in economic outcomes---, I believe that values dynamics could be incorporated in future work to explain how observed gaps between individuals can be due to differences in exposure to life events. 