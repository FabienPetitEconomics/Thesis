This thesis examines the role of inter-generational dynamics in driving distributional outcomes: chapter one describes how inter-generational conflicts over the public policy can drive labor share dynamics, chapter two explores the link between labor market polarization and social mobility, and chapter three shows the existence of spillover effects across values.
It uses a wide range of techniques by providing theoretical models to confront them to data using a broad spectrum of empirical techniques ranging from the estimation of simultaneous equations models to calibration. The key contribution lies in the conclusion that inter-generational dynamics drive distributional outcomes through several mechanisms such as conflicts, social inequality reproduction, and transmissions of values.

The two first chapters provide evidence of the link between inter-generational dynamics and the labor market. In chapter one, I argue that the inter-generational conflict over the public budget allocation has consequences for wage bargaining in the labor market and therefore for the labor share. This chapter suggests that---to understand macroeconomic dynamics in the long run---we should take into account changes in institutions that are endogenously determined by the age structure of the population. In this regard, my results provide a new conceptual framework to examine demographic dynamics and institutions in future work.

In chapter two, we show that parental income has become more important in explaining children's occupational outcomes when they enter the labor market, but also thereafter. Using British cohort data, we suggest that the structure of employment affects not only the distribution of earnings but also the degree of occupational mobility. There may be a transmission of polarization across generations, thus the increased importance of parental background may accumulate across generations creating a multiplier effect that over time that accentuates the occupational distance across groups from different backgrounds.

The second and third chapters show how intra-generational dynamics are key in understanding inter-generational ones. In chapter two, we provide a bridge between the literatures on inter- and intra-generational mobility by focusing on access to jobs at the beginning of the career and the subsequent career dynamics. Thus, we show that understanding intra-generational mobility is essential to understanding an individual’s outcome when mature. In chapter three, I argue that, within a cohort, as individuals go through different life experiences they change their values with respect to those of their peers. Those changes in values are either directly due to life events or indirectly due to spillover effects. Thus, I provide an intra-generational mechanism that is complementary to the inter-generational transmission of values.


% \printbibliography[heading=subbibliography]