Each individual belongs to a generation that was preceded by other generations and will be followed by others. Studying the dynamics of socio-economic contexts between and within those generations is crucial to academic research to understand how individuals relate to their elders, their peers, and their youngsters. This thesis presents three academic contributions to the role of inter-generational dynamics in driving distributional outcomes in several strands of the literature such as macroeconomics, labor economics, and behavioral economics.

Each generation has gone through different timelines, events, and experiences. For instance, in Western Europe, the Silent Generation (1928-1945) has known World War II, the Baby Boomers (1946-1964) were raised during the 30-year post-war boom, Generation X (1965-1980) has grown up with the first generation of personal computers, while the Millennials (1981-1996) with the Web 2.0, lastly, the Zoomers (1997-2012) were either masked or online at school due to COVID-19.
Those differences can lead to diverging perceptions of the world, hence, preferences that themselves can generate inter-generational conflicts (\citealt{meisner2021you}). My first chapter entitled ``Inter-generational conflict and the declining labor share'' argues that conflicts between generations can have consequences for distributional outcomes.

In that chapter, I emphasize the role of demographic dynamics in the allocation between capital and labor when there are conflicts between generations to determine the public budget allocation. While the labor share remained stable for decades, several OECD countries have witnessed a decline since the beginning of the 1970s (\citealt{Elsby2013Decline}, \citealt{Karabarbounis2014Global}). A heated debate among economists has emerged trying to understand the reasons for these dynamics. Yet, the literature on the labor share has paid no attention to the coincidence in timing between, on the one hand, the start of the decline of the labor share, and on the other hand, the entry of the baby-boomer cohort into adulthood, i.e. entering into the labor market and reaching voting age. 
Thus, this chapter contributes to the growing literature on the demographic determinants of technological change and automation (\citealt{Acemoglu2018Race, Acemoglu2020Robots}). I argue that the observed shift away from labor toward capital is a response to changes in labor market institutions endogenously determined by the age structure of the population, a novel mechanism that this chapter is the first to identify. 

Entering the labor market is a crucial step for every individual. Among all the differences that generations face throughout their lifecycle, the difference in labor market context is likely to be one of the most diverging.\footnote{According to OECD data, a young boomer who entered in the US labor market in 1969---at the peak of the post-war economic boom---faced a 3.51\% unemployment rate, while a young millennial who looked for a job in 2009---during the Great Recession---faced a 9.27\% unemployment rate.}
Over the last decades, the labor market has been transformed due to institutional changes (\citealt{Bentolila1990Firing}, \citealt{Bentolila2003Explaining}) and technological change, notably, with the appearance of robots and automation technologies (\citealt{Acemoglu2020Robots}). Those latter have resulted in a rise in the share of low- and high-paying jobs at the cost of middling jobs, hence, polarizing the labor market (\citealt{Autor2003Skill}, \cite{Goos2007Lousy}). 
Meanwhile, inter-generational mobility has substantially declined in countries where the polarization process has been observed; see, for example, \citet{Blanden2007Accounting} for the UK and \citet{Chetty2020Race} for the US.
My second chapter entitled ``Spreading the polarization disease: From the labor market to social mobility'' suggests that increased employment polarization may be one of the factors behind the observed decline in inter-generational mobility.

In this chapter, written with Cecilia García-Peñalosa and Tanguy van Ypersele, we use data for two British cohorts that entered the labor market at two points in time that differed considerably in terms of the structure of employment to re-examine the drivers of mobility. The data indicate that occupational changes over the individual’s career are an important source of mobility, with large shares of those in low-paying (respectively, middling) occupations moving into middling (resp. high-paying) ones. When we compare the two cohorts we find that these two sources of mobility have declined for the younger cohort and that, whatever the initial occupation, parental income has become more important in leading to occupational upgrading. Moreover, we also document that the impact of parental income increased the most in the regions where the share of middling employment fell the most. We thus bridge a gap between two literatures, one focusing on falling mobility in high-income economies and another establishing an increase in job polarization in those same countries. This chapter adds to the previous one by suggesting that the relationship between technological change and generations is two-way. On the one hand, technological change can be due to institutional changes triggered by the age structure of the population (chapter one). On the other hand, individuals all along that structure do not experience the same type of inter-generational mobility as they face different labor market contexts owing to technological change.

Between and within generations, individuals experience different patterns of social and income mobility which tend to influence their preferences for redistribution (\citealt{Piketty1995Social}, \citealt{Alesina2018Intergenerational}). However, one can think of many life experiences that can affect other types of preferences (e.g. for leisure or for fertility). Values characterize preferences as they reflect what is important in individuals' lives. Studying the dynamics of the former is key to understanding differences in preferences between economic agents which can explain differences in behavior, hence, gaps in economic outcomes. While prior work analyzes those dynamics, it only focuses on the evolution of a single value, thus, neglecting that values are inter-dependent across groups.\footnote{See, for example, \citet{Bolzendahl2004Feminist}, \citet{Cunningham2005Reciprocal}, \citet{Fernandez2007Women}, \citet{Washington2008Female}, and \citet{Grinza2017Entry}.} My third chapter entitled ``Spillover effects across values'' shows that assuming independence between values leads to underestimating to which extent life experiences affect individuals because such an assumption omits the consequences of the group membership, hence, the existence of spillover effects.

In this chapter, I argue that because group identity is defined by a cluster of values, shocks to one value that induce a change in group membership will lead to changes in other values, hence creating spillover effects. 
Those spillover effects appear when individuals seek to be consistent with the values of the group with which they identify. To elucidate the spillover effects, I build a theoretical model accounting for values consistency and endogenous group membership. Using British cohort data, I identify spillover effects through the impact of exogenous life events on values (conservatism and collectivism) in a simultaneous equations model. I also show that changes in values following life events are also associated with a change in the likelihood to vote for different political parties. My results show that individuals adjust the full set of their values when an experience occurs in their life. The findings suggest that value consistency and group identity are key drivers of values dynamics, hence, preferences. Thus, this chapter contributes to the literature on the formation of preferences in which most prior work focuses either on the inter-generational transmission (\citealt{Bisin2001Economics, Bisin2011Economics}) or the development during childhood (\citealt{Fehr2013Development}, \citealt{Doepke2017Parenting}). I, instead, provide a new mechanism---which takes place later on in individuals' lifecycle---that is based on the will to convey values that are consistent with the group with which individuals belong, hence identify. This chapter also suggests that the link between cultural values and institutions, as described by \citet{Acemoglu2021Culture}, may find its origins in the willingness of individuals to group and share consistent values.

This thesis contributes to several strands of the literature in economics (including macroeconomics, labor economics, and behavioral economics) by adopting an inter-disciplinary approach as I build on psychology and sociology, hence, bridging key literatures in social sciences. The central theme of this thesis relates to the existence of several generations embodied with individuals that are different between and within those generations due to their individuals' experiences over the lifecycle which in turn have consequences for institutions, values, and economic outcomes.

\printbibliography[heading=subbibliography]