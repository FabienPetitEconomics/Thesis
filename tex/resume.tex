Cette thèse explore le rôle des changements dans les dynamiques intergénérationnelles en économie en se concentrant sur les résultats distributionnels. 
Le premier chapitre explore le déclin de la part du travail dans un contexte de changement de structure de la population avec l’apparence des cohortes de boomer en France et aux Etats-Unis. Je soutiens que la substitution du travail par le capital opérée par les firmes est une conséquence des changements dans les institutions du marché du travail qui sont déterminées de manière endogène par la structure d’âge de la population. 
Le second chapitre décrit comment la polarisation sur le marché du travail a été accompagnée par un déclin de la mobilité sociale intergénérationnelle au Royaume-Uni. Mes co-auteurs et moi comparons deux cohortes britanniques qui sont entrées sur le marché du travail à deux moments qui diffèrent considérablement en termes de structure d’emploi. Nous trouvons que le rôle du revenu parental a augmenté pour la mobilité sociale. Nous suggérons que la compréhension des dynamiques intergénérationnelles nécessite de comprendre la mobilité intragénérationnelle. 
Le troisième chapitre examine les conséquences des événements majeurs dans la vie sur les valeurs individuelles tout au long du cycle de vie. Je développe un cadre théorique pour expliquer comment les individus ajustent leurs valeurs quand elles sont interdépendantes et affectées par des chocs dus aux événements de la vie. En utilisant des données de cohortes britanniques, je montre que les expériences de la vie changent les valeurs des individus directement mais aussi indirectement puisque les individus cherchent à être cohérents dans leurs valeurs.

\vspace{0.5cm}
\noindent\textbf{Mots clés}: Dynamiques intergénérationnelles; Part du travail; Mobilité sociale; Valeurs individuelles\\
% \textbf{Classification JEL}:
